%% Copyright (c) 2002, 2010 Sam Williams
%% Copyright (c) 2010 Richard M. Stallman
%% Permission is granted to copy, distribute and/or modify this
%% document under the terms of the GNU Free Documentation License,
%% Version 1.3 or any later version published by the Free Software
%% Foundation; with no Invariant Sections, no Front-Cover Texts, and
%% no Back-Cover Texts. A copy of the license is included in the
%% file called ``gfdl.tex''.


\chapter{Краткое путешествие по аду хакера}
\chaptermark{Краткое путешествие}

[РМС: В этой главе я только добавил несколько комментариев вроде этого.]

Ричард Столлман не мигая смотрит сквозь стекло взятого напрокат авто, ожидая зелёного света. Мы едем через центр города Кихеи на Гаваях. Едем в соседний город Пайа, где через час нас ждут несколько программистов с жёнами.

Всего пару часов назад, когда Ричард выступал в центре высокопроизводительных вычислений Мауи, город показался ему готовым к сотрудничеству. Теперь это впечатление прошло -- мы едем через сплошной одномерный пригород по широкой главной улице с магазинами, кафе, агентствами услуг. Как будто маленькая стальная клетка движется по пищеводу гигантского ленточного червя коммерции. Ещё и боковых улиц нет. Едешь только вперёд, от светофора к светофору, и эти волны потока машин кажутся перильстатикой кишечника.

Для Столлмана, жителя Восточного побережья, так медленно ехать и терять столько времени -- настоящая трагедия. [РМС: Мне тогда надо было срочно ответить на электронное письмо, да и вообще я едва успеваю делать свои дела.] Четверть мили назад можно было повернуть несколько раз и объехать затор, и это удручает сильнее всего. Но мы должны ехать за автомобилем впереди, а её водитель, местный программист, решил повести нас красивой дорогой, а не быстрой.

\enquote{Ужасно, -- вздыхает Столлман, -- почему мы не выбрали другой путь?}

В четверти мили впереди загорается зелёный. Мы продвигаемся на несколько машин вперёд и опять стоим. Так через десяток минут, под шум моторов и пронзительных гудков, добираемся до светофора. Здесь -- крупный перекрёсток и манящие выезды на загородное шоссе.

Водитель впереди и не думает поворачивать, он едет прямо через перекрёсток.

\enquote{И опять не поворачивает! -- кричит Ричард, вскидывая руки. -- Можешь в это поверить?!}

Я предпочитаю не отвечать. Я удивлён, что Столлман вообще умеет водить, и теперь, слушая по радио виолончели Yo-Yo Ma, просто любуюсь закатом в окне.

На следующем перекрёстке Ричард нервно включает поворотник для подсказки водителю впереди. Но тот всё равно едет прямо. Следующий светофор мигает за двести ярдов от нас. Столлман в ярости.

\enquote{Похоже, что он нас специально игнорирует!} -- восклицает Ричард, махая руками и жестикулируя в попытках привлечь внимание нашего гида. В его зеркале заднего вида отражается его невозмутимое лицо.

Я поворачиваюсь к Столлману и смотрю в его окно. Соседние острова Кахоолаве и Ланаи берут закат в идеальную рамку. Наверное, только местные жители не восхищаются этой красотой. Я пытаюсь обратить внимание Ричарда на неё, но тот буквально одержим водителем впереди.

Мы не поворачиваем и на следующем перекрёстке, и я сжимаю зубы. Я помню, как разработчик BSD Кит Бостик однажды предупредил меня: \enquote{Столлман терпеть не может дурацкие ходы. Если кто-то глупит, он смотрит в глаза и прямо говорит: это глупо.\hspace{0.01in}}

Я гляжу на водителя впереди и понимаю, что Столлмана убивает именно глупость его решения.

\enquote{Как можно выбирать путь, даже не задумываясь, насколько он эффективен?} -- ворчит Ричард.

Слово \enquote{неэффективность} отравляет атмосферу. Мало что раздражает ум хакера больше, чем неэффективность. Именно неэффективность программы проверки лазерного принтера Xerox заставила Столлмана искать её исходный код. Именно неэффективность переписывания кода, присвоенного коммерческими компаниями, заставила Столлмана дать бой Symbolics и основать проект GNU. Если, как выразился Жан Поль Сартр, ад это другие люди, то ад хакера состоит из их ошибок и глупостей. И Столлман всю свою жизнь спасает человечество от этой пылающей бездны.

Мрачная метафора становится ещё очевиднее, если посмотреть на город вокруг. Это даже не город, а беспорядочное, бессмысленное нагромождение зданий, парковок и через раз горящих фонарей. Словно плохо спроектированная компьютерная программа. Здесь нужна продуманная сеть дорог, а не одна широкая улица, постоянно забитая машинами. Видеть такую нелепость хакеру неприятно до дрожи, как нам слышать какой-нибудь противный скрежет или визг.

\enquote{Несовершенные системы бесят хакеров, -- заметил Стивен Леви ранее, -- поэтому они не любят водить машину. Из-за всех этих хаотично настроенных светофоров и нелепых односторонних улиц появляются задержки, которые чертовски \textit{ненужны}, и хакеру навязчиво хочется переставить знаки, снести светофоры и вообще переделать всю транспортную сеть. }\endnote{Steven Levy, \textit{Hackers} (Penguin USA, 1984): 40.}

Но Столлмана больше бесит нерешительность нашего проводника. Вместо того, чтобы найти эффективный путь, как интуитивно сделал бы любой хакер, он решил играть по правилам градостроителей. Подобно Вергилию Данте, он проводит полную экскурсию по этому аду, хотим мы этого или нет.

Я хочу поделиться со Столлманом своими мыслями, и тут водитель впереди наконец-то включает правый поворотник, и мы поворачиваем к шоссе Пилани. Сгорбленные плечи Ричарда расслабляются, напряжение в машине спадает. Внезапно мы упираемся в знаки \enquote{Дорожные работы} и кучи земли. Поперёк улицы стоит бульдозер. Шоссе Пилани на четверть мили дальше.

Столлман смотрит на это всё широко раскрытыми глазами, словно не понимая что случилось. Только когда наш гид начинает неуклюже разворачиваться перед нами, Ричард закипает.

\enquote{За что?! -- стонет он, откидывая голову на подголовник, -- он что, не знал, что дорога перекрыта? Он был не в курсе, что тут ремонт? Он что, специально это делает?!} [РМС: Я имел в виду, что он намеренно хотел ехать подольше. Цитаты довольно неточные.]

Гид, наконец, разворачивается и на обратном пути к главной улице пропускает нас вперёд, при этом качает головой и извиняется, улыбаясь. Его жесты выдают небольшое огорчение, изрядно перекрытое фатализмом островитянина. Он словно говорит: \enquote{Ребята, это Мауи, ну что поделать!}

Терпение Столлмана лопается.

\enquote{Какого хрена ты улыбаешься?! -- кричит он в стекло, -- это твоя долбаная ошибка! Если бы поехали моим путём, было бы намного лучше!} [РМС: Это наверняка неточные цитаты, потому что я так не выражаюсь. Это было не интервью, и у Вильямса не было диктофона. В целом, всё было как описано, но тон моих слов преувеличен.]

На словах \enquote{моим путём} Ричард дважды дёргает руль, как психующий ребёнок на игрушечном авто. В его голосе страдания мешаются с гневом, и кажется, что он вот-вот заплачет.

Однако слёз не будет. Подобно молнии, эта вспышка кончается так же резко, как начинается. Тихо вздыхая, Столлман переключает передачу и разворачивается. Когда мы въезжаем на главную улицу, его лицо такое же спокойное, как полчаса назад, когда мы покидали отель.

Скоро мы выезжаем на перекрёсток и сворачиваем на загородное шоссе. Ярко-жёлтое солнце перемещается с левого плеча Ричарда в зеркало заднего вида. Оно красит в красно-оранжевый цвет деревья, пролетающие за окнами по обе стороны.

Следующие двадцать минут мы слышим только шум мотора, ветра за окном, и виолончелей Yo-Yo Ma.

\theendnotes
\setcounter{endnote}{0}
