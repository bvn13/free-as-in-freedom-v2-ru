\thispagestyle{plain}

\noindent Это \enquote{Освобождение вашего компьютера (2.0): Ричард Столлман и революция свободного программного обеспечения} -- русский перевод книги \textit{Free as in Freedom 2.0: Richard Stallman and the Free Software Revolution}, исправленной версии \textit{Free as in Freedom: Richard Stallman's Crusade for Free Software}.

\bigskip

\noindent \copyright{} Сэм Вильямс, 2002, 2010\\
\copyright{} Ричард М. Столлман, 2010\\

Разрешается копировать, раздавать и/или редактировать эту книгу в соответствии с условиями лицензии свободной документации GNU или GNU FDL версии 1.3 или более поздних версий, опубликованных фондом свободного ПО -- без неизменяемых разделов и текста на обложке. Текст лицензии находится в приложении под названием \enquote{Лицензия GNU для свободно используемой документации}.

\bigskip

\noindent Опубликовано фондом свободного ПО\\
Франклин-стрит 51 , этаж 5\\
Бостон, Массачусетс 02110-1335\\
США\\
ISBN: 9780983159216\\

\bigskip

\noindent Автор фотографии PDP-10 в главе 7 -- Родни Брукс. В образе святителя ИГНУциуса Ричарда Столлмана в главе 8 сфотографировал Стиан Эйкеланн.

\bigskip

Перевод на русский язык и вёрстку выполняли\endnote{Список сформирован автоматически на основе git log.} \input{translators}

Исходный код доступен в репозитории
\href{https://code.dumpstack.io/etc/free-as-in-freedom-v2-ru}{code.dumpstack.io/etc/free-as-in-freedom-v2-ru}.

Если вы заметили опечатки либо неточности -- отправляйте исправления по адресу
\href{mailto:patch@dumpstack.io}{patch@dumpstack.io}.

Перевод версии \input{build-ver} сборка от \input{build-time}

\theendnotes
\setcounter{endnote}{0}
